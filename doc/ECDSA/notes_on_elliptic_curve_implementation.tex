\documentclass{article}
%\usepackage{authblk}
\usepackage{amssymb, amsmath}
\usepackage{hyperref}
\usepackage{verbatim}

%\author[1]{Todor Milev}
\date{April 2018}
%\affil[1]{FA Enterprise System}
\title{Elliptic curve secp256k1 \\ Implementation notes}
\author{Todor Milev\footnote{FA Enterprise System}\\ todor@fa.biz}
\newcommand{\secpTwoFiveSixKone}{{\bf secp256k1}}
\renewcommand{\mod}{{~\bf mod~}}
\begin{document}
\maketitle
\section{Introdution}
Public/private key cryptography is arguably the most important aspect of modern crypto-currency systems. The somewhat slow execution of private/public key cryptography algorithms appears to be one of the main bottlenecks of FAB's Kanban system. 


Following Bitcoin, FAB coin uses the standard public/private key cryptography ECDSA over \secpTwoFiveSixKone. Here, ECDSA stands for Elliptic Curve Digital Signature Algorithm and \secpTwoFiveSixKone{} stands for the elliptic curve:

\[
y^2 = x^3 + 7
\]
(we specify the base point later), over the finite field:

\[
\mathbb Z / p\mathbb Z, 
\]
where
\begin{equation}\label{eqThePrime}
p= 2^{256} - 2^{32} - 977.
\end{equation}

In this document, we discuss and document technical details of FAB's implementation of ECDSA over  \secpTwoFiveSixKone{}. Our openCL implementation is based on the project \cite{secp256k1:openCLimplementationHanh0}, which is in turn based on the C project libsecp256k1 \cite{Wuille:secp256k1}.

\section{Operations in $\mathbb Z / p\mathbb Z$}
Recall from (\ref{eqThePrime}) that $p$ is the prime given by
\[
p= 2^{256} - 2^{32} - 977.
\]
In this section, we describe our implementation of $Z / p\mathbb Z$.
\subsection{Representations of numbers}
A number in $x$ in $\mathbb Z / p\mathbb Z$ is represented by a large integer $X$, in turn represented by a sequence of $10$ small integers $x_0,\dots, x_9$ for which $0 \leq x_i < 2^{32}$ and such that
\[
X = \sum_{i=0}^{9} x_i \left(2^{26}\right)^i.
\]
The representations of $x$ is not unique but becomes so when we request that 
\[
0\leq x_i < 2^{26}
\]
and 
\[
0 \leq X < p = 2^{256} - 2^{32} - 977.
\]
We say that the unique representation $x_0, \dots, x_9$ of $x$ above is its \emph{normal form} (and $x$ is \emph{normalized}). Two elements of $ \mathbb Z / p \mathbb Z$ are equal if and only if their normal forms are equal. We will not assume that a number $x$ is represented by its normal form as some of the operations described below do not require that.

In what follows, we shall use the notation
\[
a'=a \mod q
\]
to denote remainder $0\leq a' <q$  of $a$ when dividing $a$ by $q$. For the rest of this section, we also set 
\[
d= 2^{26}.
\]
In this way the normal form of a large number $X$ is its $d$-base representation (here $d$ is the ``digit'' of the base).

\subsection{Computing the normal form of $x$}\label{secNormalFormOfFieldElement}

\subsection{Multiplying two elements}
Let $a$ be an element represented by $A = \sum a_i d^i$ with $a_0, \dots, a_8<2^{30}$ and $a_9<2^{22}$. Likewise, let $b$ be represented by $B=\sum b_j d^j$ with analogous inequalities on the coefficients $b_j$. In this section, we show how to compute the normal form of $a\cdot b$. This operation is implemented in the function \verb|ECMultiplyFieldElementsInner|. 

Compute as follows.
\[
\begin{array}{rcl}
\displaystyle A\cdot B &=&\displaystyle \left(\sum_{i=0}^9 a_i d^i \right) \left(\sum_{j=0}^9 b_j d^j\right) \\
&=&\displaystyle  \sum_{k=0}^{18} \left(\underbrace{ \sum_{i=0}^k a_i b_{k-i} }_{=\bar t_k} \right) d^k \\
&=&\displaystyle \sum_{k=0}^{18} \bar t^k d^k,
\end{array}
\]
where we have set $\displaystyle \bar t_k = \left(\sum_{i=0}^k a_i b_{k-i} \right)$. Since $0\leq a_i,b_i < d$, the $\bar t_k$'s are smaller than $2^{64}$ and can be computed by standard arithmetic using the \verb|uint64_t| integer type of the GPU. 

Let
\[
A\cdot B = \sum_{k=0}^{19} t_k d^k
\] 
with $0\leq t_k < d$ be the unique representation of $A\cdot B$ base $d$. Then the $t_k$ can be computed from the $\bar t_k$'s consecutively via 
\[
\begin{array}{rcl}
\displaystyle t_0 &= &\displaystyle \bar t_0 \mod d\\
\displaystyle c_0 &= &\displaystyle \left\lfloor \frac{\bar t_0}{d} \right\rfloor\\ \\\hline \\
\displaystyle t_1 &= &\displaystyle c_0 + \bar t_0 \mod d\\
\displaystyle c_1 &= &\displaystyle \left\lfloor \frac{c_0 + \bar t_0}{d} \right\rfloor \\
\displaystyle t_2 &= &\displaystyle c_1 + \bar t_1 \mod d\\
\displaystyle c_2 &= &\displaystyle \left\lfloor \frac{c_1 + \bar t_1}{d} \right\rfloor \\
&\vdots&\\
\displaystyle t_{18} &= &\displaystyle c_{17} + \bar t_{17} \mod d\\
\displaystyle c_{18} &= &\displaystyle \left\lfloor \frac{c_{17} + \bar t_{17}}{d} \right\rfloor \\ \\ \hline
\displaystyle t_{19} &= &\displaystyle  c_{18}.
\end{array}
\]
In the above, all computations except the first and the last are similar, as indicated by the horizontal line, and $\lfloor\bullet \rfloor$ stands as usual for the floor function. Both $\mod d$ and division by $d$ are carried out by bit-shift operations and are therefore fast.

Let $S = (A\cdot B \mod p)$; computing the normal form $S = \sum s_i d^i$, $0\leq s_i < d$, $S < p$ is the final aim of the present discussion. Since $2^{256} \mod p = 2^{32}+977$, it follows that 
\begin{equation} \label{eqReducet0plust10dPower10}
\begin{array}{rcll}
t_0+ t_{10} d^{10} \mod p 
&=&\displaystyle t_0+t_{10} 2^{26\cdot 10} &\mod p\\
&=&\displaystyle t_0 + t_{10} \cdot 2^4\cdot 2^{256}  &\mod p\\
&=&\displaystyle t_0 + t_{10} 2^4 \left(2^{32} + 977 \right)&\mod p\\
&=&\displaystyle \left(t_0 +t_{10}2^4\cdot 977\right) + t_{10} 2^{10}\cdot d&\mod p \\
\displaystyle \text{set }g_{0} = t_0 +t_{10}2^4\cdot 977 \mod d \\
\displaystyle \text{set }f_{0} =\left\lfloor \frac{t_0 +t_{10}2^4\cdot 977}{d}\right\rfloor \\
&=& g_0 + \left(\underbrace{ f_0+t_{10} 2^{10}}_{=h_0}\right) d &\mod p.
\end{array}
\end{equation}
Set $\displaystyle h_0 = f_0+t_{10} 2^{10}$ as indicated above. The computation above shows how ``reduce'' the $d$-digit $t_{10}$ by modifying the two least significant $d$-digits. Accounting for the ``carry-over'' digit $h_0$, we can continue with this process for the next pair of digits $ t_1 d+t_{11} d^{11}$, and so on. This is done below (similar steps have been omitted).
\[
\begin{array}{rcll}
h_0 d +t_1 d+ t_{11} d^{11} \mod p 
&=&\displaystyle \left(h_0+t_1+t_{11} d^{10}\right) d &\mod p\\
&=& \dots \text{compute as in (\ref{eqReducet0plust10dPower10})} \dots\\
\displaystyle \text{set }g_{1} = h_0+t_1 +t_{11}2^4\cdot 977 \mod d \\
\displaystyle \text{set }f_{1} =\left\lfloor \frac{h_0+t_1 +t_{11}2^4\cdot 977}{d}\right\rfloor \\
&=& g_1d + \left(\underbrace{ f_1+t_{11} 2^{10}}_{=h_1}\right) d^2 &\mod p\\
&\vdots&\\
h_8 d^9 + t_{9} d^9 + t_{19} d^{19}\mod p &=& \dots\\
\displaystyle \text{set }g_{9} = h_8+t_9 +t_{19}2^4\cdot 977 \mod d \\
\displaystyle \text{set }f_{9} =\left\lfloor \frac{h_8+t_9 +t_{19}2^4\cdot 977}{d}\right\rfloor \\
&=&g_9 d^9 + \left( \underbrace{f_9+t_{19} 2^{10}}_{=h_9}\right)d^{10} &\mod p.
\end{array}
\]
In the computations above, $t_{19}$ is maximum $26$ bits long and so
\begin{equation}\label{eqh9inequality}
h_9 < f_9 + 2^{26}2^{10} < 2^{37}.
\end{equation}
In order to obtain the normal form of $S$, we need to modify the digits $g_9$ and $h_9$ as follows.
\[
\begin{array}{rcll}
\displaystyle g_9 d^9 + h_9 d^{10} \mod p &=&\displaystyle  (g_9+h_9 d )d^9 &\mod p \\
\displaystyle \text{set }r =g_9 + h_9 d \mod 2^{22}\\
\displaystyle \text{set }m = \left\lfloor \frac{g_9 + h_9 d }{2^{22}} \right\rfloor\\
&=& \left( r + m \cdot 2^{22} \right) d^9  &\mod p\\
&=& r d^9 + m 2^{256} &\mod p\\
&=& r d^9 + m\left(2^{32}+977\right) &\mod p.\\
\end{array}
\]
From inequality (\ref{eqh9inequality}), we get that $h_9 d < 2^{37} 2^{26} = 2^{63}$, and so the computations above fit in \verb|uint64_t| type. Again using (\ref{eqh9inequality}) we get the following estimates for the sizes of digits.
\[
\begin{array}{rcl}

\displaystyle g_9+ h_9 d &<& \displaystyle 2^{64}\\
\displaystyle m=\left\lfloor \frac{g_9 + h_9 d }{2^{22}} \right\rfloor&<& \displaystyle \frac{2^{64}}{2^{22}} = 2^{42}\\
m\left(2^{32}+977\right) &<& 2^{33}\cdot 2^{42} = 2^{75} < d^3.
\end{array}
\]
Thus the number $m\left(2^{32}+977\right)$ can be written in the form $z_0 + z_1 d + z_2 d^2 $ with $0\leq z_0, z_1, z_2 < d$. Collecting the information so far, we get 
\begin{equation}\label{eqAtimesBalmostNormalForm}
A\cdot B \mod p = (z_0 + g_0) + (z_1 + g_1)d + (z_2+g_2)d^2 + \sum _{k=3}^{8} g_k d^k + r d^9,
\end{equation}
where $z_0+g_0 < 2d$, $z_1+g_1 <2d$, $z_2+g_2<2d$, $g_k< d$, $r < 2^{22}$. Except in the cases when $z_i+g_i\geq d$, this expression is normalized. To ensure our result is normalized, we need to reduce the representation above according to Section \ref{secNormalFormOfFieldElement}. This does not need to be done immediately as the form \eqref{eqAtimesBalmostNormalForm} satisfies all assumptions of the present discussion. 





\bibliographystyle{plain}
\bibliography{../bibliography.bib}
\end{document}